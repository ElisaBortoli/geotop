%=============================================================%
\chapter{Snow and Glacier}
%=============================================================%

\section{The snow physical processes}

The snow properties (height, density, mechanics, composition) change with time, according to the meteorological conditions. We may divide the snow changes into four classes: i) snow fall, that results in snow deposition through precipitation; ii) snow erosion/accumulation, that is the results of the action of the wind (blowing snow); iii) snow metamorphism, which is the morphological variation as a result of the thermodynamic conditions of the snowpack and iv) snow melting, that results in the ultimate melting of the snow.


\subsection{Snow fall}

\subsection{Snow erosion/accumulation (blowing snow)}

\subsection{Snow metamorphism}\label{met}


%On the modeling point of view, the metamorphism result in a variation in thickness and density of snow layers, which ultimately result in a variation in time of SWE and U.

%Every snowfall deposits its snow layer on the ground, but the duration of its persistence varies according to the climate. If the snowpack remains only a few days and then is depleted away due to climatic conditions, it is referred to as \emph{temporary} or \emph{ephemeral snow cover}. This is typical when snowfall occurs at lower altitudes or at higher altitudes early in autumn or late in spring. In winter, if the climatic conditions do not allow melt of the deposited snow, snow cover accumulates, remains for a longer duration, and can persist until spring or summer. Such a snow cover is categorized as \emph{seasonal snow cover}. Its time evolution is separated in an \emph{accumulation period}, when the snowpack energy balance is negative, i.e. the snowpack loses energy and its temperature is decreasing, and in a \emph{melt period}, characterized by a positive energy balance so that the snowpack increases its temperature and begins to melt. Commonly, it is possible to observe a little melt during the accumulation period, and little accumulation during the melt period. Furthermore, a loss of snow mass can also occur through sublimation (i.e. the direct state change from solid ice to water vapour) or, to a much lesser degree, through evaporation (from liquid water in the snow) during both the accumulation and the melt periods. Snow sublimation, evaporation and melt together are referred to as snow ablation. The contribution of evaporation to snow ablation is normally negligible. If the snow cover is not completely melted away in the summer season and persists until the following accumulation period, it is referred to as \emph{permanent snow cover}, which gradually turns into glaciers \citep{Singh01}.
%A snow cover has a stratified structure because snowfall is deposited in a series of layers. Each layer has its own physical properties, determined by the snow conditions at the time of deposition and the following changes in the snow structure and properties. The density of newly fallen snow is determined by the configuration of the snowflakes, which is a function of air temperature, wind speed and the physical conditions occurring during their formation in the precipitating cloud \citep{Mellor64}. Observed values of newly fallen snow densities can range from 4 to 340 kg m$^{-3}$ \citep{Mckay70}, with lower values occurring under calm and very cold conditions and higher values under higher winds and higher temperature. Actually, higher wind speeds tend to break snowflakes and to pack them into denser layers \citep{Dingman1994}. However, newly fallen snow density usually ranges from 70 to 150 kg m$^{-3}$ \citep{Dingman1994}. 

The process dealing with morphological changes in snow structure is known as snow metamorphism. Four mechanisms are responsible for this process: (1) destructive metamorphism, (2) constructive metamorphism, (3) melt metamorphism, and (4) pressure metamorphism \citep{Alford74}.

\paragraph{destructive metamorphism}
Snow crystals have a branched structure when they are formed in the precipitating cloud. \emph{Desctructive metamorphism} occurs when branched crystals break down after reaching the ground, either through the mechanical forces of wind or through thermodynamic stress \citep{Colbeck83}. As a consequence, grains become rounded and grain packing of the snowpack occurs, causing an increase in snow density. This process is primarily important immediately after snow has fallen, when density increases  at about 1$\%$ per hour, on average \citep{Gunn65}. The process ceases to be important when snow density reaches about 250 kg m$^{-3}$ \citep{Anderson76}.

\paragraph{Constructive metamorphism} It occurs as a result of the extensive transfer of mass in the vapour phase. Two conditions can be distinguished: when nearly uniform temperature prevails in the snowpack and when a temperature gradient is established in the snowpack \citep{Singh01}. In the first case, the water vapor flux takes place, as the saturated vapor pressure $e_s$ over a curved surface increases with the curvature of the surface.
%, as shown in the following expression \citep{Singh01}:
%\begin{equation}\label{ecurv}
%e_s=s_{s0}\left(1+\frac{a}{Tr}\right),
%\end{equation}
%where $e_{s0}$ is the saturated vapour pressure (mbar) over the flat ice surface at temperature $T$ (K), $a$ is a thermodynamic parameter (mbar K$^{-1}$ m$^{-1}$), and $r$ is the radius of curvature (m) of the surface, positive if the surface is convex towards the ice, and negative if the surface is concave. 
So water vapor tends to sublime from smaller grains and to redeposit on nearby larger grains, having less curved surfaces, so larger grains grow at the expenses of smaller grains \citep{Dingman1994}. In this way, the grains can reach a size up to 1 mm \citep{Sommer70}. In addition, owing to close grain packing, adjacent grains can gradually build a neck at the points of contact. Necks are further fed by water vapor deposition, as the surface at the necks is usually concave and is associated with lower vapor pressure than plane surface.
%as expressed by (\ref{ecurv}).
This process is referred to as \emph{sintering} and causes further densification and snow strengthening \citep{Singh01}. On the contrary, if a temperature gradient is established, water vapor mainly flows in the direction opposed to the gradient. This process results in an intensive growth of favorably oriented grains, while other grains are reduced or disappear altogether, causing a decrease of the number of crystals per unit volume and an increase in the dimension of snow grains \citep{Singh01} to significantly more than 1 mm \citep{Alford74}. A relatively shallow snowpack under very cold air produces a strong upward-decreasing temperature gradient within the snow, as temperature near the ground is close to  0 $^\circ$C due to heat insulating power of snow. Under this conditions, snow at the base of the pack evaporates at a high rate, producing a basal layer of characteristic large planar crystals with low density (200-300 kg m$^{-3}$) and low strength (as the crystals are weakly bonded) called \emph{depth hoar} \citep{Singh01}.

\paragraph{Melt metamorphism} It refers to the presence of liquid water in the snowpack and increases the rate at which grains become rounded and results in growth of larger particles and disappearance of smaller particles, because grains melt first where the surface curvature is lower. As a consequence, a melting snowpack is typically an aggregation of rounded grains the size of up to about 2 mm \citep{Colbeck78}. In addition, refreezing of meltwater or rain causes grains to join together in clusters the size up to 15 mm \citep{Alford74}. Melting and refreezing of a large amount of liquid water may form very high density layers, referred to as ice layers or lenses, representing a rapid transition from snow to ice. The melt water accelerates packing by lubricating the grains and permits very close packing since the surface tension of a liquid water film tends to pull the grains together \citep{Singh01}. So, in wet snow, higher density is produced than in dry snow. 

\paragraph{Pressure metamorphism} is the result of shifting and rotation of individual grains caused by the weight of overlying snow layers. Then, a more closely packed structure is produced. On glaciers, the pressure of thick layers of accumulated snow is the principal factor leading to the formation of solid ice. Pressure metamorphism, however, proceeds slowly and it may take many years to convert the initial low density snow to ice \citep{Alford74}.

\vspace{0.5cm}
\noindent Except for the temporary formation of depth hoar, all the processes of metamorphism result in a progressive increase in snow density during the accumulation season.

\subsection{Snowmelt}

At the beginning of the melt period, the snowpack normally shows a vertical stratification with several layers of markedly contrasting density. During melting, snow density keeps on increasing, but the vertical dishomogeneity tends to disappear. In the melt season, snow density may fluctuate on an hourly scale, as the formation, the drainage, and the refreezing of melt water take place \citep{Dingman1994}.
Snowmelt occurs when the snowpack energy input is more or less continually positive and is usually divided into three different phases \citep{Dingman1994}:
\begin{itemize}
  \item \emph{Warming phase}, during which the snowpack warms until it is isothermal at 0 $^\circ$C; 
  \item \emph{Ripening phase}, during which melting takes place, but the snowpack completely retain the meltwater, until, at the end of this phase, the snowpack cannot retain any more liquid and is said to be \emph{ripe};
  \item \emph{Output phase}, during which melting occurs and results in water outflow.
\end{itemize}
However, this subdivision is a simplification of how snowmelt progresses. Actually, at the air interface, snow normally presents a marked daily variation, which is damped out deeper in the snowpack. Therefore, melting can already occur at the snow surface before the ripening phase, when solar radiation is high enough during the daytime, but the meltwater produced percolates into the deeper layers, which are still cold, and refreezes, thus releasing latent heat which warms the deeper snow. Similarly, in the output phase, the temperature at the surface normally falls below 0$^\circ$C at night so that warming is needed for melting to continue on the following day. 


\section{Snow definitions}
Snow is a granular porous medium consisting of ice and pore spaces. When the snow temperature is below 0 $^\circ$C, the pores are filled only with air, and water vapour and snow is referred to as dry snow; on the contrary, when the snow temperature is equal to 0 $^\circ$C, the pores contain also liquid water, and snow is referred to as wet snow. 

Snow can be described by a simplified mixture theory \citep{Morris87}. On a spatial scale of centimeters, the media approach a continuum and can be described by bulk properties. The volume of the snow ($V_{sn}$) is composed by the volume of ice ($V_i$), water ($V_w$) and gaseous ($V_g$) components, e.g. air and water vapor. 
\begin{equation}\label{eq:Vsn}
V_{sn}:=V_i + V_w + V_{g}
\end{equation}
Let us define $V_{\phi}:=V_w+V_g$ as the volume in the total medium where the fluid may percolate. One obtains:
\begin{equation}
V_{sn}=V_i + V_{\phi}
\end{equation}
Dividing (\ref{eq:Vsn}) by $V_{sn}$ one obtains the volumetric fractions:
\begin{equation}
\theta_i + \theta_w + \theta_{g}=1
\end{equation}
Let us define the snow porosity $\phi$ (-) as the volumetric fraction of $V_{\phi}$. One obtains:
\begin{equation}
\phi=\theta_w+\theta_g=1-\theta_i
\end{equation}
%
The mass of snow $M_{sn}$ stocked in $V_{sn}$, excluding the mass of vapor and other gaseous constituents, is represented by the mass of water $M_w$ (kg) and the mass of ice $M_i$ (kg). 
\begin{equation}\label{eq:Msn}
M_{sn}:=M_i + M_w=\rho_i V_i + \rho_w V_w
\end{equation}
Let us define the snow density ($\varrho$) as the ratio between the mass of snow $M_{sn}$ and the volume $V_{sn}$: the snow density results therefore in a weighted mean of water and ice contents:
%the  is the density of the snow layer ``$l$'' (hereafter, for simplicity, we will call the snow density as $\varrho_l$).
%The snow density of a layer ``l'' is calculated as a weighted mean of the water and ice contents:

\begin{equation}\label{eq:rho}
\varrho:=\frac{M_{sn}}{V_{sn}}= \rho_i \theta_i + \rho_w \theta_w 
\end{equation}

\noindent Let us define HS (mm) as the height of snow (which coincides with the height of $V_{sn}$) to the soil surface, and the snow water equivalent (SWE) (mm) as the water content of $V_{sn}$ in equivalent mm of water:

\begin{equation}\label{eq:swe1}
\textrm{SWE}:= \textrm{HS} \cdot \frac{\overline \varrho}{\rho_w}
\end{equation}

\noindent where $\overline \varrho$ is the mean density of the whole snow mantle. \\
Generally the snow density is very heterogeneous and so would be better to consider different layers of snow.
Let us discretize the snow volume into $N_{sn}$ layers of snow, the first being the one in contact with the soil, and the last one ($N_{sn}$) being in contact with the atmosphere. The snow height may be calculated as:
%
\begin{equation}
\textrm{HS}:= \sum_{l=1}^{N_{sn}} \Delta z_l
\end{equation}
and, putting (\ref{eq:rho}) into (\ref{eq:swe1}), the SWE of the layer ``l'' becomes:
\begin{equation}\label{eq:swe2}
\textrm{SWE}_l=\frac{\varrho_l}{\rho_w} \cdot  \Delta z_l = \Delta z_l \cdot \left(\theta_w+ \frac{\rho_i}{\rho_w}  \theta_i  \right)_l
\end{equation}
and the SWE is eventually the sum of all SWE$_l$:
\begin{equation}
\textrm{SWE}= \sum_{i=1}^{N_{sn}} \textrm{SWE}_l
\end{equation}


\subsection{Snow mass balance}
The mass balance in $V_{sn}$ results:
\begin{equation}\label{eq:wb0}
\frac{\partial}{\partial t} M_{sn} + V_{sn} \ \rho_w \ \frac{\partial J^m}{\partial z} + V_{sn} \ \rho_i \ \frac{\partial I}{\partial z}=0
\end{equation}
Considering (\ref{eq:Msn}), excluding the flux of ice ($I=0$) and dividing by $V_{sn}$:
\begin{equation}\label{eq:wb1}
%\frac{\partial}{\partial t} \left(\frac{M_{sn}}{V_{sn}}\right) + \rho_w \ \frac{\partial J^m}{\partial z} =0
\frac{\rho_i}{\rho_w} \frac{\partial \theta_i}{\partial t} + \frac{\partial \theta_w}{\partial t}+ \ \frac{\partial J^m}{\partial z} =0
\end{equation}
%
%Both the snow volume $V_{sn}$ and the snow mass $M_{sn}$ vary in time according to the metamorphism accumulation/melting processes. The derivative in time becomes:
%\begin{eqnarray}\label{eq:wb_adiuv}
%\nonumber \frac{\partial}{\partial t} \left(\frac{M_{sn}}{V_{sn}}\right)=
%\frac{1}{V_{sn}} \cdot \frac{\partial M_{sn}}{\partial t}\bigg{|}_{V_{sn}} -
%\frac{M_{sn}}{V_{sn}^2} \cdot \frac{\partial V_{sn}}{\partial t}\bigg{|}_{M_{sn}} =\\
%\rho_i \ \frac{\partial \theta_i}{\partial t}\bigg{|}_{\textrm{HS}} + \rho_w \frac{\partial \theta_w}{\partial t}\bigg{|}_{\textrm{HS}} - 
%\left( \rho_i \theta_i + \rho_w \theta_w \right) \cdot  \frac{1}{\textrm{HS}} \cdot \frac{\textrm{HS}}{\partial t}\bigg{|}_{\textrm{SWE}}
%\end{eqnarray}
%
Integrating in z on a layer ``l'', considering the layer thickness constant:
\begin{equation}\label{}
\Delta z_l \left(\frac{\rho_i}{\rho_w} \frac{\partial \theta_i}{\partial t} + \frac{\partial \theta_w}{\partial t} \right)+ J_l^m -J_{l-1}^m =0
\end{equation}
%
that, according to (\ref{eq:swe2}), becomes:
\begin{equation}\label{}
\frac{\partial}{\partial t}\textrm{SWE}_l+ J_l^m -J_{l-1}^m =0
\end{equation}

\noindent $J^m$ (m s$^{-1}$) is the water flux at the boundaries that, according to the Darcy's law, results:

\begin{equation}
J^m =-K_{sn} \cdot \frac{\partial}{\partial z} \left(\frac{P}{\rho_w \ g} + z_f \right)
\end{equation}

\noindent where $z_f$ is the height of the snow layer with respect to a fixed reference and $K_{sn}$ (m s$^{-1}$) is the snow hydraulic conductivity. The water flux $J^m$, considering just the gravitational contribution (P=0): 

\begin{equation} \label{eq:J}
J^m =-K_{sn}  \ \cos \alpha
\end{equation}

\noindent the hydraulic conductivity is usually given by the ratio between the permeability $k_l$ (m$^2$) and the dynamic viscosity $\mu_l$ (kg m$^{-1}$ s$^{-1}$) of the liquid:

\begin{equation}
K_{sn}=\frac{k_l}{\mu_l} \cdot \rho_w \cdot g
\end{equation}

\noindent \citet{Colbeck72} related $k_{l}$ to the intrinsic permeability of the snow matrix $k_s$ (m$^2$) and to the effective water saturation $S_e$ (-) by means of this expression \citep{Brooks1964}:

\begin{equation}
k_l=k_s \cdot S_e^3
\end{equation}

\noindent where $k_s$ is \citep{Shimizu70}:

\begin{equation}
k_s=0.077 \cdot d^2 \cdot \exp \left(-7.8 \ \frac{\varrho}{\rho_w}\right)
\end{equation}


\begin{table}[htdp]
\begin{center}
\begin{tabular}{|c|c|c|}
 \hline
   snow density (kg m$^{-3}$) & snow grain diameter (m) & \\
   \hline
	50 & 5 $\cdot$ $10^{-5}$ & fresh snow\\
	100 & 1 $\cdot$ $10^{-4}$ & fresh snow\\
	200 & 2 $\cdot$ $10^{-4}$ & fresh snow\\
	300 & 5 $\cdot$ $10^{-4}$ & fine-grained old snow\\
	400 & 8 $\cdot$ $10^{-4}$ & fine-grained old snow\\
	500 & 1.2 $\cdot$ $10^{-3}$ & coarse-grained old snow\\
	600 & 2 $\cdot$ $10^{-3}$ & melting snow\\
	700 & 3 $\cdot$ $10^{-3}$ & melting snow\\
	\hline
\end{tabular}
\end{center}
\caption{\it{Values of snow density correlated with values of snow grain diamater, deduced from qualitative considerations from \citet{SNTHERM}.  }} \label{grainsize}
\end{table}%


\noindent $d$ (m) is the snow grain diameter that is correlated to the snow density  \citep{SNTHERM}, and $S_e$ is the water saturation level of the snow:

\begin{equation}
S_e=\frac{\theta_w-S_r \cdot \phi}{\phi - S_r \cdot \phi}
\end{equation}

\noindent where $S_r$ (-) is the irreducible water saturation \citep{Colbeck72} and accounts for the capillarity of the snow, and $\phi$ (-) is the snow porosity.
Finally, Eq. (\ref{eq:J}) becomes:

\begin{equation} \label{}
J_z^m =-\frac{k_{s} \cdot S_e^3}{\mu_l} \cdot \rho_w \cdot g \cdot \cos \alpha
\end{equation}


%\noindent where:

%\begin{equation}
%W_i:= \rho_w \left(\Delta z \theta_w \right)_i
%\end{equation}
%\noindent $W_i$ (kg m$^{-2}$ s$^{-1}$) is the mass of water subject to phase change.

\subsection{Snow energy balance}

The 1D energy conservation equation:
% &&&
\begin{equation}\label{eq:eb1}
\frac{\partial U}{\partial t} + \frac{\partial}{\partial z} \left[G+J^e \right]=0 
\end{equation}
%
\begin{itemize}
\item U [J m$^{-3}$] is the internal energy:

\begin{equation}\label{u1}
U=C \left(T-T_{ref}\right) + L_f \rho_{w} \theta_{w},
\end{equation}
where $T$ is the snow temperature, $T_{ref}=0^\circ$C is the reference temperature,  $C_{sn}$ is the volumetric thermal capacity of the snow (J m$^{-3}$ K$^{-1}$), calculated as a weighted mean of the heat capacities of ice and water:
%\noindent L'energia interna della neve $U$ si definisce in rieferimento alla temperatura di 0$^\circ$C e si scrive:
\begin{equation}
C=\rho_i c_i \theta_i + \rho_w c_w \theta_w.
\end{equation}

%I calori specifici del ghiaccio $c_i$ e dell'acqua $c_w$ valgono rispettivamente 2117 J kg$^{-1}$ K$^{-1}$ e 4188 J kg$^{-1}$ K$^{-1}$, mentre il calore latente di fusione dell'acqua $L_f$ vale 3.337$\cdot10^6$ J kg$^{-1}$. 

\item $G$ [W m$^{-2}$] is the conduction heat flux according to Fourier law:
% &&&
\begin{equation}\label{}
G=- \lambda \cdot  \frac{\partial T}{\partial z}
\end{equation}
%
\noindent where $\lambda$ [W m$^{-1}$ K$^{-1}$] is the snow thermal conductivity {\bf (Sturm, 1977)}:
% &&&
\begin{equation}\label{}
\lambda= 0.138 - 1.01 \cdot \varrho + 3.23 \cdot \varrho^2
\end{equation}
%



\item $J^e$ is the advection energy flux provided by the flowing water:
% &&&
\begin{equation}\label{}
J^e=\rho_w \left[L_f + c_w \ T \right]
\end{equation}
%

\noindent Integrating (\ref{eq:eb1}) on a snow layer ``l'' of depth $\Delta z_l$ one obtains:

\begin{equation}\label{eq:eb2}
\Delta z_l \frac{\partial U_l}{\partial t}+ \left(G_l - G_{l-1} + J_l^e -J_{l-1}^e \right) =0
\end{equation}

\end{itemize}






\section{Splitting method}
The snow varies in height and water equivalent according to the physical process under consideration. In general, we may assume that the physical processes are a combination of three activities: (i) phase change, (ii) water fluxes, and (iii) compression. 
Let us indicate with the superscript $^{fl}$ the changes due to the ``flux'' of water, and the superscript $^{ph}$ refers to the changes due to the ``phase change'' of water. 
According to this scheme, Eq. (\ref{eq:wb1}) 
%modified according to (\ref{eq:wb_adiuv}), 
becomes:
%
%\begin{equation}\label{eq:wb20}
%\rho_i \ \frac{\partial \theta_i}{\partial t}\bigg{|}_{\textrm{HS}} + \rho_w \frac{\partial \theta_w^{ph}}{\partial t}\bigg{|}_{\textrm{HS}}+ \rho_w \frac{\partial \theta_w^{fl}}{\partial t}\bigg{|}_{\textrm{HS}} - 
% \ \frac{\left( \rho_i \theta_i + \rho_w \theta_w\right)}{\textrm{HS}} \cdot \frac{\partial \textrm{HS}}{\partial t}\bigg{|}_{\textrm{SWE}} + \rho_w \ \frac{\partial J^m}{\partial z} =0
%\end{equation}
%
\begin{equation}\label{eq:wb20}
\rho_i \frac{\partial \theta_i}{\partial t} + \rho_w \frac{\partial \theta_w^{ph}}{\partial t}= -\rho_w \left(\frac{\partial \theta_w^{fl}}{\partial t}+ \frac{\partial J^m}{\partial z}\right) 
\end{equation}
%
At the same manner, Eq. (\ref{eq:eb1}) may be arranged as:
\begin{equation}\label{eq:eb20}
\frac{\partial U^{ph}}{\partial t} + \frac{\partial G}{\partial z}+\frac{\partial U^{fl}}{\partial t} +\frac{\partial J^e}{\partial z}=0
\end{equation}
%
Equations (\ref{eq:wb20}) and (\ref{eq:eb20}) are equivalent to the system:
%\begin{equation} \label{}
%\left\{
%\begin{array}{ll|}
%\rho_i \ \frac{\partial \theta_i}{\partial t}\bigg{|}_{\textrm{HS}} + \rho_w \frac{\partial \theta_w^{ph}}{\partial t}\bigg{|}_{\textrm{HS}}=0\\
%\\
%\frac{\partial U^{ph}}{\partial t} + \frac{\partial G}{\partial z} =0\\
%\\
%\frac{\partial \theta_w^{fl}}{\partial t}\bigg{|}_{\textrm{HS}} + \rho_w \ \frac{\partial J^m}{\partial z} =0\\
%\\
%\frac{\partial U^{fl}}{\partial t} +\frac{\partial J^e}{\partial z}=0\\
%\\
%\frac{\left( \rho_i \theta_i + \rho_w \theta_w\right)}{\textrm{HS}} \cdot \frac{\partial \textrm{HS}}{\partial t}\bigg{|}_{\textrm{SWE}} =0
%\end{array} \right. \,
%\end{equation}
\begin{equation} \label{}
\left\{
\begin{array}{ll|}
\rho_i \frac{\partial \theta_i}{\partial t} =- \rho_w \frac{\partial \theta_w^{ph}}{\partial t}\\
\\
\frac{\partial U^{ph}}{\partial t} + \frac{\partial G}{\partial z} =0\\
\\
\frac{\partial \theta_w^{fl}}{\partial t}+ \frac{\partial J^m}{\partial z}=0\\
\\
\frac{\partial U^{fl}}{\partial t} +\frac{\partial J^e}{\partial z}=0\\
\end{array} \right. \,
\end{equation}

\noindent The above system may be solved in the following steps:
\begin{enumerate}
\item {\bf phase change}: during this step one can determine the mass of water/ice subject to phase change. Let us assume that no volume expansion during freezing is allowed, i.e. the density of water $\rho_w$ and ice $\rho_i$ are equal; furthermore, during this step we assume no water flux is allowed, i.e. the volume $V_{sn}$ is constant. The equations are:
\begin{equation}\label{eq:bbb1}
\frac{\partial \theta_i}{\partial t} =- \frac{\partial \theta_w^{ph}}{\partial t}
\end{equation}
\begin{equation}\label{eq:eb12}
\frac{\partial U^{ph}}{\partial t} + \frac{\partial G}{\partial z} =0
\end{equation}
%
Eq. (\ref{eq:eb12}) may be solved according to \citet{dall2010energy}.

\item {\bf water infiltration}
\begin{equation}\label{}
\frac{\partial \theta_w^{fl}}{\partial t} + \rho_w \ \frac{\partial J^m}{\partial z} =0
\end{equation}

\item {\bf energy advection}
\begin{equation}\label{}
\frac{\partial U^{fl}}{\partial t} + \frac{\partial J^e}{\partial z}=0
\end{equation}

\end{enumerate}


%and applying Eq. (\ref{eq:Msn}) one obtains:
%\begin{equation}\label{eq:wb1ff}
%\frac{\partial}{\partial t} \left( \theta_w + \frac{\rho_i}{\rho_w} \theta_i \right)+ \frac{\partial J^m}{\partial z} =0
%\end{equation}

%and applying Eq. (\ref{eq:Msn}) one obtains:
%\begin{equation}\label{}
%\frac{\partial}{\partial t} \left( \theta_w + \frac{\rho_i}{\rho_w} \theta_i \right)+ \frac{\partial J^m}{\partial z} =0
%\end{equation}

%\noindent Integrating (\ref{eq:wb1ff}) on a snow layer ``l'' of depth $\Delta z_l$ one obtains:

%\begin{equation}\label{eq:wb2ff}
%\Delta z_l \left(\frac{\partial \theta_w}{\partial t}\right)_l+ \frac{\rho_i}{\rho_w} \Delta z_l \left(\frac{\partial \theta_i}{\partial t}\right)_l + \left( J_l^m -J_{l-1}^m \right) =0
%\end{equation}

%\noindent Putting (\ref{eq:swe2}) into (\ref{eq:wb2ff}), the water balance of a snow layer becomes:

%\begin{equation}\label{eq:wb3ff}
%\frac{\partial}{\partial t} \textrm{SWE}_l+ \left( J_l^m -J_{l-1}^m \right) =0
%\end{equation}




\section{Boundary conditions}

On the modeling point of view, the time evolution of snow cover coincides in a variation in thickness and density of snow layers, as a consequence of the variation in time of SWE and U.






\begin{equation}
\frac{\partial}{\partial t} \left( \theta_w + \frac{\rho_w}{\rho_i} \theta_i \right)+ \frac{\partial}{\partial z} \left[J_a^m - J_s^m \right] =0
\end{equation}

where the superscript ``m'' stands for mass and the subscripts ``a'' and ``s'' stand respectively for the upper ``atmosphere'' and the lower ``soil'' boundaries.
% &&&
\begin{equation}\label{Eq:cons_en}
\frac{\partial U}{\partial t} + \frac{\partial }{\partial z} \left[ G_a - G_s + J_a^e - J_s^e\right]=0  \hspace{2cm} [W \ m^{-3}]
\end{equation}
%




where the superscript ``m'' stands for mass and the subscripts ``a'' and ``s'' stand respectively for the upper ``atmosphere'' and the lower ``soil'' boundaries.


Consideriamo lo strato $i$-esimo e integriamo la (\ref{Eq:cons_en}) lungo $z$. Dal teorema della divergenza si ottiene:
% &&&
\begin{equation}\label{Eq:En_lay_i}
\frac{\partial U_i}{\partial t} + \frac{G_{i+1/2}-G_{i-1/2}}{\Delta z_i} =0
\end{equation}
%
dove con $G$ si intende ora la componente $G_z$ lungo $z$. Discretizzando secondo uno schema alle differenze finite unidimensionale lungo la coordinata $z$, positiva verso il basso, $G_{i+1/2}$ vale:
\begin{equation}\label{}
G_{i+1/2}=-\lambda_{i+1/2} \frac{T_{i+1}-T_{i-1}}{z_{i+1}-z_i}
\end{equation}
dove $T$ rappresenta la temperatura dello strato e $z$ la coordinata del centro dello strato.
La condizione al contorno superiore con l'atmosfera, secondo la (\ref{BilEn}), vale:
% &&&
\begin{equation}\label{Eq:cond_sup}
\frac{\partial U_s}{\partial t}\Delta z_s+ G_{s+1/2} = G_{sn}(T_s)= R_n(T_s)+P-H(T_s)-L(T_s)
\end{equation}
% 
dove con il pedice $s$ si intende la superficie della neve, coincidente con lo strato superficiale.
La condizione al contorno inferiore (sotto l'ultimo strato di suolo) vale:
% &&&
\begin{equation}\label{Eq:cond_inf}
\frac{\partial U_i}{\partial t}\Delta z_i- G_{i-1/2} =G_{geo} \hspace{2cm} i=N_s+N_g
\end{equation}
dove $G_{geo}$ \`e il flusso di calore proveniente dal basso ed \`e un parametro di ingresso.


La definizione di $U$ in (\ref{u1} tiene conto solo del cambiamento di fase dovuto al bilancio di energia (f), perci\`o deve essere integrata con gli effetti della percolazione dell'acqua o dell'ingresso di acqua derivante da nuova neve o da pioggia (p), oltre che dal compattamento (c) della neve che risulta in una diminuzione di porosit\`a. Risulta dunque:

\begin{equation}
d\theta_w=(d\theta_w)_{f}+(d\theta_w)_{p}+(d\theta_w)_{c},
\end{equation}

\noindent dove $(d\theta_w)_{p}$ \`e dovuto al cambiamento di fase, $(d\theta_w)_{f}$ al flusso d'acqua e $(d\theta_w)_{c}$ alla compattazione della neve. In forma differenziale la variazione di energia interna diventa:

\begin{equation}\label{Eq:dU}
dU=C_{sn} dT + L_f \rho_w (d\theta_w)_f
\end{equation}

\noindent Combinando la (\ref{Eq:En_lay_i}) con la (\ref{Eq:dU}) si ottiene:

\begin{equation}\label{Eq:fin_En}
\Delta z_i \ C_{sn} \frac{T_i^{n+1}-T_i^n}{\Delta t} + L_f \frac{\Delta W_i}{\Delta t} + G_{i+1/2}-G_{i-1/2}=0
\end{equation}

\noindent dove $\Delta W_i:=\rho_w \ \Delta z_i \ d(\theta_w)_f$ [kg m$^{-2}$] viene definita come la massa d'acqua soggetta a cambiamento di fase per metro quadrato nello strato $i$-esimo. 
%L'energia interna \`e relativa allo stato di riferimento costituito dall'acqua allo stato solido alla temperatura di 0$^\circ$C. 
%Se U$>0$, il manto nevoso \`e in equilibrio con un certo contenuto di acqua liquida, mentre se U$<0$ \`e presente soltanto acqua allo stato solido e pertanto U pu\`o essere utilizzata per calcolare la temperatura media dello strato di neve $T_{sn}$.
Appare chiaro ora come, per lo strato $i$-esimo, il termine $\Delta z_i \ C_{sn}(T_i^{n+1}-T_i^n)/\Delta t$ equivalga a $\Delta Q_s$ della (\ref{BilEn1}), pari al cambiamento di temperatura nello strato di neve nell'intervallo di tempo, mentre il termine $L_f \Delta W_i / \Delta t$ equivalga a $\Delta Q_m$, pari allo scioglimento della neve.\\
Le Equazioni (\ref{Eq:fin_En}), (\ref{Eq:cond_sup}) e (\ref{Eq:cond_inf}) rappresentano un sistema non lineare di equazioni differenziali alle derivate parziali che deve essere risolto attraverso opportuni metodi iterativi.
