\chapter{1D: domain definition and characterization}



As pointed out in Fig. \ref{Fig_sim_flowchart}, the 1D simulation may defined in two ways: 
\begin{enumerate}
\item with maps: in this case the user must provide also the topographical maps together with the land cover, the soil type and, if present, the initial conditions maps. Furthermore, the user must give in input also the coordinates of the simulation points (see Fig. \ref{grid3Dversante_points} and \ref{Fig_meteoST1}). The model automatically extrapolates the information on the give points through the provided maps;
\item without maps: in this case, the user must provide all the necessary information about the topography, land cover and soil type of the simulation points.
\end{enumerate}

\noindent In both cases the domain discretization along the Z coordinate (Fig. \ref{Fig_discr3d} on the right) must be properly defined as described in Table \ref{domain_Zcoord}.

\begin{center}
\begin{longtable}{|p {3.5 cm}|p {4.7 cm}|p {1. cm}|p{0.8 cm}|p{1.4 cm}|p{0.8 cm}|p{1.3 cm}|}
\hline
\textbf{Keyword} & \textbf{Description} & \textbf{M. U.} & \textbf{range} & \textbf{Default Value} & \textbf{Sca / Vec} & \textbf{Str / Num / Opt} \\ \hline
\endfirsthead
\hline
\multicolumn{7}{| c |}{continued from previous page} \\
\hline
\textbf{Keyword} & \textbf{Description} & \textbf{M. U.} & \textbf{range} & \textbf{Default Value} & \textbf{Sca / Vec} & \textbf{Str / Num / Opt} \\ \hline
\endhead
\hline
\multicolumn{7}{| c |}{{continued on next page}}\\ 
\hline
\endfoot
\endlastfoot
\hline
%PointSoilType & Soil type of the simulation point & - &  & NA & vec & num \\ \hline
SoilLayerThicknesses & vector defining the thickness of the various soil layers. If not present, a column of 5 layers 100 mm thick will be assumed & mm &  & 100 & vec & num \\ \hline
SoilLayerNumber & number of soil layers (is calculated after the number of components of the vector SoilLayerNumber) & - &  & 5 & sca & num \\ \hline
\caption{Keywords of parameters referred to soil layer}
\label{domain_Zcoord}
\end{longtable}
\end{center}

\section{Without maps}

\subsubsection{Parameters}
\begin{center}
\begin{longtable}{|p {5.5 cm}|p {3. cm}|p {1.2 cm}|p{1.0 cm}|p{1.2 cm}|p{0.8 cm}|p{1. cm}|}
\hline
\textbf{Keyword} & \textbf{Description} & \textbf{M. U.} & \textbf{range} & \textbf{Default Value} & \textbf{Sca / Vec} & \textbf{Log / Num} \\ \hline
\endfirsthead
\hline
\multicolumn{7}{| c |}{continued from previous page} \\
\hline
\textbf{Keyword} & \textbf{Description} & \textbf{M. U.} & \textbf{range} & \textbf{Default Value} & \textbf{Sca / Vec} & \textbf{Log / Num} \\ \hline
\endhead
\hline
\multicolumn{7}{| c |}{{continued on next page}}\\ 
\hline
\endfoot
\endlastfoot
\hline
PointLandCoverType \index{PointLandCoverType} & Land Cover type of the simulation point & - &  & NA & vec & num \\ \hline
PointSoilType \index{PointSoilType} & Soil type of the simulation point & - &  & NA & vec & num \\ \hline
PointElevation \index{PointElevation} & elevation of the point of simulation & m a.s.l. &  & NA & vec & num \\ \hline
PointSlope \index{PointSlope} & Slope steepness of the simulation point & degree &  & NA & vec & num \\ \hline
PointAspect \index{PointAspect} & Aspect of the simulation point & degree &  & NA & vec & num \\ \hline
PointSkyViewFactor \index{PointSkyViewFactor}& Sky View Factor of the simulation point & - &  & NA & vec & num \\ \hline
PointCurvatureNorthSouthDirection \index{PointCurvatureNorthSouthDirection} & N-S curvature of the simulation point & m$^{-1}$ &  & NA & vec & num \\ \hline
PointCurvatureWestEastDirection \index{PointCurvatureWestEastDirection} & W-E curvature of the simulation point & m$^{-1}$ &  & NA & vec & num \\ \hline
PointCurvatureNorthwestSoutheastDirection \index{PointCurvatureNorthwestSoutheastDirection}& N-W curvature of the simulation point & m$^{-1}$ &  & NA & vec & num \\ \hline
PointCurvatureNortheastSouthwestDirection \index{PointCurvatureNortheastSouthwestDirection} & N-E curvature of the simulation point & m$^{-1}$ &  & NA & vec & num \\ \hline
PointDrainageLateralDistance \index{PointDrainageLateralDistance} & Lateral Drainage distance of the simulation point & m &  & NA & vec & num \\ \hline
PointLatitude \index{PointLatitude} & Latitude of the simulation point & degree &  & NA & vec & num \\ \hline
PointLongitude \index{PointLongitude} & Longitude of the simulation point & degree &  & NA & vec & num \\ \hline
PointHorizon \index{PointHorizon} & number of the HorizonPointFile that describes the horizon of the simulation point & - &  & NA & vec & num \\ \hline
\caption {Keywords of topographical, land cover and soil type characteristics that may be set in geotop.inpts. Each parameter may be give in input as a vector, each component representing a point. Otherwise the characteristics may be summarized in the file PointFile, each value corresponding to the proper header defined in Table \ref{headers_topo_par1D}. }
\label{topo_par1d_topo}
\end{longtable}
\end{center}


\subsubsection{Files}

\begin{center}
\begin{longtable}{|p {2.5 cm}|p {8.5 cm}|p {1 cm}|p {1 cm}|}
\hline
\textbf{Keyword} & \textbf{Description}  \\ \hline
\endfirsthead
\hline
\multicolumn{4}{| c |}{continued from previous page} \\
\hline
\textbf{Keyword} & \textbf{Description}  \\ \hline
\endhead
\hline
\multicolumn{4}{| c |}{{continued on next page}}\\ 
\hline
\endfoot
\endlastfoot
\hline
PointFile \index{PointFile} & name of the file providing the properties for the simulation point  \\ \hline
HorizonPointFile \index{HorizonPointFile} & name of the file providing the horizon of the simulation point  \\ \hline
\caption{Keywords of files related to soil/rock spatial characterization for 1D simulation}
\label{key1D_data}
\end{longtable}
\end{center}

\subsubsection{Headers}

\begin{center}
\begin{longtable}{|p {6.5 cm}|p {6 cm}|p {2 cm}|p {1 cm}|}
\hline
\textbf{Keyword} & \textbf{Description} & \textbf{Associated file}  \\ \hline
\endfirsthead
\hline
\multicolumn{4}{| c |}{continued from previous page} \\
\hline
\textbf{Keyword} & \textbf{Description} & \textbf{Associated file}  \\ \hline
\endhead
\hline
\multicolumn{4}{| c |}{{continued on next page}}\\ 
\hline
\endfoot
\endlastfoot
\hline
HeaderHorizonAngle \index{HeaderHorizonAngle} & String representing the header of the column HorizonAngle of the HorizonPoint and HorizonMeteoStation files & HorizonPoint / HorizonMeteoStation  \\ \hline
HeaderHorizonHeight \index{HeaderHorizonHeight} & String representing the header of the column HorizonHeight of the HorizonPoint and HorizonMeteoStation files & HorizonPoint / HorizonMeteoStation  \\ \hline
HeaderPointElevation \index{HeaderPointElevation} & column name in the file PointFile for the elevation of the point & PointFile  \\ \hline
HeaderPointSlope \index{HeaderPointSlope} & column name in the file PointFile for the slope steepness of the point & PointFile  \\ \hline
HeaderPointAspect \index{HeaderPointAspect} & column name in the file PointFile for the aspect of the point & PointFile  \\ \hline
HeaderPointSkyViewFactor \index{HeaderPointSkyViewFactor} & column name in the file PointFile for the sky view factor of the point & PointFile  \\ \hline
HeaderPointCurvatureNorthSouthDirection \index{HeaderPointCurvatureNorthSouthDirection} & column name in the file PointFile for the N-S curvature of the point & PointFile  \\ \hline
HeaderPointCurvatureWestEastDirection \index{HeaderPointCurvatureWestEastDirection} & column name in the file PointFile for the E-W curvature of the point & PointFile  \\ \hline
HeaderPointCurvatureNorthwestSoutheastDirection \index{HeaderPointCurvatureNorthwestSoutheastDirection} & column name in the file PointFile for the NW-SE curvature of the point & PointFile  \\ \hline
HeaderPointCurvatureNortheastSouthwestDirection \index{HeaderPointCurvatureNortheastSouthwestDirection} & column name in the file PointFile for the NE-SW curvature of the point & PointFile  \\ \hline
HeaderPointDrainageLateralDistance \index{HeaderPointDrainageLateralDistance} & column name in the file PointFile for the distance of lateral drainage & PointFile  \\ \hline
HeaderPointHorizon \index{HeaderPointHorizon} & column name in the file PointFile that provides the number of the HorizonPointFile that describes the horizon of the simulation point & PointFile  \\ \hline
HeaderPointLatitude \index{HeaderPointLatitude} & column name in the file PointFile for the latitude of the point & PointFile  \\ \hline
HeaderPointLongitude \index{HeaderPointLongitude} & column name in the file PointFile for the longitude of the point & PointFile  \\ \hline
HeaderPointID \index{HeaderPointID} & column name in the file PointFile for the identification ID of the point & PointFile  \\ \hline
HeaderCoordinatePointX \index{HeaderCoordinatePointX} & column name in the file PointFile for the x coordinate of the point & PointFile  \\ \hline
HeaderCoordinatePointY \index{HeaderCoordinatePointY} & column name in the file PointFile for the y coordinate of the point & PointFile  \\ \hline
\caption{Keywords of headers that specify the soil/rock spatial characterization for 1D simulation}
\label{headers_topo_par1D}
\end{longtable}
\end{center}






\section{With maps}

\subsubsection{Maps}

\begin{center}
\begin{longtable}{|p {3.5 cm}|p {6.5 cm}|p {3 cm}|p {4 cm}|}
\hline
\textbf{Keyword} & \textbf{Description}  \\ \hline
\endfirsthead
\hline
\multicolumn{4}{| c |}{continued from previous page} \\
\hline
\textbf{Keyword} & \textbf{Description}  \\ \hline
\endhead
\hline
\multicolumn{4}{| c |}{{continued on next page}}\\ 
\hline
\endfoot
\endlastfoot
\hline
DemFile \index{DemFile} & name of the file providing the DEM map  \\ \hline
SkyViewFactorMapFile \index{SkyViewFactorMapFile}& name of the file providing the sky view factor map  \\ \hline
SlopeMapFile \index{SlopeMapFile} & name of the file providing the slope steepness map  \\ \hline
RiverNetwork \index{RiverNetwork} & name of the file providing the river network map  \\ \hline
AspectMapFile \index{AspectMapFile} & name of the file providing the aspect map  \\ \hline
CurvaturesMapFile \index{CurvaturesMapFile} & name of the file providing the curvature map  \\ \hline
%BedrockDepthMapFile & name of the file providing the bedrock depth map  \\ \hline
LandCoverMapFile \index{LandCoverMapFile} & name of the file providing the land cover map  \\ \hline
SoilMapFile \index{SoilMapFile} & name of the file providing the soil map  \\ \hline
\caption{Keywords of input file related to the domain}
\label{Input_top_1D_withoutmaps}
\end{longtable}
\end{center}

\subsubsection{Files}

\begin{center}
\begin{longtable}{|p {2.5 cm}|p {8.5 cm}|p {1 cm}|p {1 cm}|}
\hline
\textbf{Keyword} & \textbf{Description}  \\ \hline
\endfirsthead
\hline
\multicolumn{4}{| c |}{continued from previous page} \\
\hline
\textbf{Keyword} & \textbf{Description}  \\ \hline
\endhead
\hline
\multicolumn{4}{| c |}{{continued on next page}}\\ 
\hline
\endfoot
\endlastfoot
\hline
PointFile \index{PointFile} & name of the file providing the properties for the simulation point  \\ \hline
\caption{Keyword of the file related to the spatial characterization of soil/rock properties. The parameters identified by the row index represent the value corresponding to the SoilMapFile map.}
\label{key3D_data_ii}
\end{longtable}
\end{center}


\subsubsection{Parameters}
\begin{center}
\begin{longtable}{|p {2.2 cm}|p {4.2 cm}|p {2.7 cm}|p{1.0 cm}|p{1.2 cm}|p{0.8 cm}|p{1. cm}|}
\hline
\textbf{Keyword} & \textbf{Description} & \textbf{M. U.} & \textbf{range} & \textbf{Default Value} & \textbf{Sca / Vec} & \textbf{Log / Num} \\ \hline
\endfirsthead
\hline
\multicolumn{7}{| c |}{continued from previous page} \\
\hline
\textbf{Keyword} & \textbf{Description} & \textbf{M. U.} & \textbf{range} & \textbf{Default Value} & \textbf{Sca / Vec} & \textbf{Log / Num} \\ \hline
\endhead
\hline
\multicolumn{7}{| c |}{{continued on next page}}\\ 
\hline
\endfoot
\endlastfoot
\hline
PointID \index{PointID}& identification code for the point of simulation &  &  & NA & vec & num \\ \hline
CoordinatePointX \index{CoordinatePointX}& coordinate X if PixelCoordinates is 1, number of row of the matrix if PixelCoordinates is 0 & m (according to the geographical projection of the maps) &  & NA & vec & num \\ \hline
CoordinatePointY \index{CoordinatePointY} & coordinate Y if PixelCoordinates is 1, number of column of the matrix if PixelCoordinates is 1 & m (according to the geographical projection of the maps) &  & NA & vec & num \\ \hline
Latitude \index{Latitude} & Average latitude of the basin, positive means north, negative means south & degree & -90, 90 & 45 & sca & num \\ \hline
Longitude \index{Longitude} & Average longitude of the basin, eastwards from 0 meridiane & degree & 0, 180 & 0 & sca & num \\ \hline
\caption{Keywords of point characterization for the choice of points where to perform a 1D simulation}
\label{topo_par1d_withmaps}
\end{longtable}
\end{center}

\subsubsection{Headers}

\begin{center}
\begin{longtable}{|p {4.15 cm}|p {8 cm}|p {2 cm}|p {1 cm}|}
\hline
\textbf{Keyword} & \textbf{Description} & \textbf{Associated file}  \\ \hline
\endfirsthead
\hline
\multicolumn{4}{| c |}{continued from previous page} \\
\hline
\textbf{Keyword} & \textbf{Description} & \textbf{Associated file}  \\ \hline
\endhead
\hline
\multicolumn{4}{| c |}{{continued on next page}}\\ 
\hline
\endfoot
\endlastfoot
\hline
HeaderPointID \index{HeaderPointID}& column name in the file PointFile for the identification ID of the point & PointFile  \\ \hline
HeaderCoordinatePointX \index{HeaderCoordinatePointX} & column name in the file PointFile for the x coordinate of the point & PointFile  \\ \hline
HeaderCoordinatePointY \index{HeaderCoordinatePointY}& column name in the file PointFile for the y coordinate of the point & PointFile  \\ \hline
\caption{Keywords of headers that specify the soil/rock spatial characterization for 1D simulation}
\label{headers_topo_par1D}
\end{longtable}
\end{center}


